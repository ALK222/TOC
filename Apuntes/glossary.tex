\newabbreviation{rom}{ROM}{Read Only Memory}
\newabbreviation{ieee}{IEEE}{Institute of Electrical and Electronics Engineers}
\newabbreviation{hdl}{HDL}{Hardware Description Language}
\newabbreviation{fpga}{FPGA}{Field-Programmable Gate Array}
\newabbreviation{sta}{STA}{Static Time Analisis}
\newabbreviation{uf}{UF}{Unidad Funcional}
\newabbreviation{hw}{HW}{Hardware}
\newabbreviation{pipo}{PIPO}{Parallel-Input Parallel-Output}
\newabbreviation{sipo}{SIPO}{Serial-Input Parallel-Output}
\newabbreviation{piso}{PISO}{Parallel-Input Serial-Output}
\newabbreviation{siso}{SISO}{Serial-Input Serial-Output}
\newabbreviation{asm}{ASM}{Algorithmic State Machine}
\newabbreviation{cpu}{CPU}{Central Proccesing Unit}
\newabbreviation{sram}{SRAM}{Static RAM}
\newabbreviation{dram}{DRAM}{Dynamic RAM}
\newabbreviation{sdram}{SDRAM}{Synchronous DRAM}
\newabbreviation{eeprom}{EEPROM}{Electrically Erasable Programmable ROM}
\newabbreviation{ram}{RAM}{Random Access Memory}
\newglossaryentry{vhdl}{
	name=VHDL,
	description=Lenguaje de especificación hardware definido por el \gls{ieee} utilizado para describir circuitos digitales y la automatización de diseño electrónico. VHDL es el acrónimo que surge tras combinar los acrónimos vhsic y \gls{hdl}.
}
